\section{A coarsening framework based on HARP}\label{sec:harp}

The sequence \( G_0, G_1, G_2, \dots, G_L \) introduced in Section~\ref{sec:performance-complexity} is generated in HARP one step at a time, that is, graph \( G_i \) is generated from the graph \( G_{i - 1} \) by coarsening it -- lowering the number of nodes and edges while preserving in some sense the general structure of the graph. Following \cite{chen_harp_2018}, let \( \varphi_i \) denote the mapping \( G_i = \varphi_i \left( G_{i - 1} \right) \). In this work, we restrict the definition of such a coarsening \( \varphi_i \) to only consist of a series of edge contractions. Therefore, each coarsening corresponds to a set \( \mathcal{C} \subseteq E \left( G \right) \), i.e. a subset of edges of the original graph. This description of a graph coarsening is inspired by \cite{schulz_mining_2019}, which is why it differs from the weaker definition of a graph reduction as selecting a subset of nodes and edges \cite{huang_scaling_2021,loukas_graph_2019}.

In an overview, the HARP algorithm consists of the following steps:

\begin{enumerate}
  \item \textbf{Dataset augmentation}. The graph \( G \) is consecutively reduced in size by the application of several graph coarsening schemas.
\end{enumerate}
After all the coarsened graphs are pre-computed ahead-of-time, the method itself can be executed by repeating the following steps on the graphs from the coarsest to the finest (i.e. from \( G_L \) to \( G_0 \)):
\begin{enumerate}\setcounter{enumi}{1}
  \item \textbf{Training on an intermediary graph}. The graph embedding model is trained on \( G_i \), producing \( \Phi_{G_i} \), an embedding of the graph in a Euclidean space.
  \item \textbf{Embedding prolongation}. The embedding \( \Phi_{G_i} \) is refined into \( \Phi_{G_{i - 1}} \), which is then used as the starting point for training on \( G_{i - 1} \).
\end{enumerate}
These steps are repeated until \( \Phi_{G_0} \) is computed. The particular details of the coarsening and prolongation steps are further explained in \cite{chen_harp_2018}.
