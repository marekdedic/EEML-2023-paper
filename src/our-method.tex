\section{HARP extension for flexible performance-complexity balancing}\label{sec:our-method}

Graph representation learning methods such as node2vec typically have a large number of parameters -- on the widely used OGBN-ArXiv dataset (see \cite{hu_open_2021}), the state-of-the-art node2vec model has over 21 million parameters. At the same time, recent works in the domain of graph learning have started to focus more heavily on simpler methods as a competitive alternative to heavy-weight ones (see \cite{frasca_sign_2020, huang_combining_2020}). As the authors of \cite{chen_harp_2018} observed, HARP improves the performance of models when fewer labelled data are available. The proposed lower complexity models based on HARP could also improve performance in a setting where only low fidelity data are available for large parts of the graph. Coarser models could be trained on them, with a subsequent training of finer models using only a limited sample of high fidelity data.

In this work, we extend the general HARP framework to study the preformance-complexity characteristics of graph data. To this end, we propose alternatives to both the coarsening as well as the prolongation step of HARP. First, in Section~\ref{sec:adaptive-prolongation}, we replace the simple prolongation approach by an adaptive prolongation algorithm. Second, in Section~\ref{sec:coarsening-algorithms}, we study two alternative ways of coarsening the graph.

\subsection{Granularity of the HARP algorithm}

In standard HARP, once the coarsened graphs are obtained, the way to train the graph embedding is fairly straightforward. Starting with the coarsest graph, an embedding model such as node2vec is trained. Following that, a step to a graph that is one level finer is made. The embedding learned on the immediately preceding coarser graph is \name{prolonged} to the embedding of the following finer graph, in which the representations of merged nodes are copied and reused.
 Then, with this prolonged embedding as the starting state, the embedding algorithm continues training and this process is repeated until reaching the original graph.

While this style of prolongation is fine when HARP is used only as a means of pre-training, this approach is far too crude when studying the relationship between graph complexity and the quality of graph embedding and subsequent downstream applications. For example, the widely-used Cora dataset \cite{yang_revisiting_2016} has in its original form 2708 nodes, while the graph resulting from one application of the HARP coarsening schema has only about 1100 nodes (exact numbers may differ run-by-run). Such a relatively high reduction ratio effectively prevents any sufficient understanding of the relationship between graph reduction and changes in the quality of its embedding.

In order to offer a more fine-grained observation of the graph complexity and its effect on the downstream task, we present the adaptive prolongation approach. This algorithm works with the pre-coarsened graphs produced for example by HARP, however, the embedding is learned in a different manner.

\subsection{The adaptive prolongation approach}\label{sec:adaptive-prolongation}

\begin{figure*}
  \centering
  \includegraphics[width=\textwidth]{images/adaptive-prolongation/adaptive-prolongation.pdf}
    \caption{A schematic explanation of the adaptive prolongation algorithm for obtaining the embedding \( \Psi_{i} \) from \( \Psi_{i + 1} \).}
  \label{fig:adaptive-prolongation}
\end{figure*}

The adaptive prolongation approach aims to replace the fixed steps defined by the used coarsening algorithm (such as HARP) by a variable number of smaller \enquote{micro-steps}, each of a predefined size that can be chosen independently from the underlying coarsening and its step size. Moreover, the prolongation procedure is driven by the interplay of the downstream task with the local properties of the underlying graph. This enables the method to produce embeddings with different level of granularity in different parts of the graph, e.g. an embedding that is coarse inside clusters of similar nodes and at the same time fine at the border between such clusters.

Let us denote \( \Psi_K, \dots, \Psi_0 \) the embedding sequence produced by the adaptive prolongation approach. To achieve the desired finer control over granularity the sequence \( \Psi_K, \dots, \Psi_0 \) needs to be decoupled from the graph sequence \( G_L, \dots, G_0 \), which in particular implies that the value of \( K \) is independent of \( L \). This is in contrast to the standard HARP prolongation, where the sequence of embeddings \( \Phi_{G_L}, \dots, \Phi_{G_0} \) is directly tied to the previously obtained graphs.

Similarly to standard HARP prolongation, the algorithm starts with the coarsest graph \( G_L \), trains a graph model to compute its embedding \( \Psi_K \) and gradually refines it until reaching the embedding \( \Psi_0 \) of \( G_0 \), or, alternatively, until a stopping criterion is met, as outlined in Section~\ref{sec:performance-complexity}. Instead of directly setting the value of \( K \), a fixed number of nodes \( n_p \) is prolonged in each step. These prolongation steps are interlaid with continued training of the graph model, as in standard HARP.  A description of a single prolongation step from \( \Psi_{i + 1} \) to \( \Psi_i \) follows, is schematically outlined in Figure~\ref{fig:adaptive-prolongation} and available as pseudocode in the supplementary material.

The procedure keeps track of all the edge contractions that were made in the dataset augmentation part of the algorithm and gradually reverses them. To this end, apart from the embedding \( \Psi_i \), the set of all contractions yet to be reversed as of step \( i \) is kept as \( \mathcal{C}_L^{(i)}, \dots, \mathcal{C}_0^{(i)} \), with the initial values \( \mathcal{C}_j^{(K)} \) corresponding to the underlying coarsening \( \varphi_j \) as defined in Section~\ref{sec:harp}.

In each prolongation step, the embedding \( \Psi_{i + 1} \) is prolonged to \( \Psi_i \) by selecting a set of \( n_p \) contractions \( \mathcal{C}_\mathrm{prolong} \) from the original coarsening procedure and undoing them by copying and reusing the embedding of the node resulting from the contraction to both of the nodes that were contracted. To obtain this set of contractions, nodes of \( G_0 \) are first ordered in such a way that corresponds to the usefulness of prolonging them. Subsequently, the set \( \mathcal{C}_\mathrm{prolong} \) is selected from \( \mathcal{C}_L^{(i + 1)}, \dots, \mathcal{C}_0^{(i + 1)} \) by selecting contractions affecting nodes in the aforementioned order, until \( n_p \) contractions are selected. If multiple contractions affecting the same node are available in the sequence \( \mathcal{C}_L^{(i + 1)}, \dots, \mathcal{C}_0^{(i + 1)} \), one is selected from \( \mathcal{C}_j^{(i + 1)} \) corresponding to the coarsest-level coarsening, that is, from \( \mathcal{C}_j^{(i + 1)} \) with the highest possible \( j \). The sequence \( \mathcal{C}_L^{(i)}, \dots, \mathcal{C}_0^{(i)} \) is produced from \( \mathcal{C}_L^{(i + 1)}, \dots, \mathcal{C}_0^{(i + 1)} \) by removing all of the edges contained in \( \mathcal{C}_\mathrm{prolong} \) (each edge from \( \mathcal{C}_\mathrm{prolong} \) will be contained in exactly one of \( \mathcal{C}_L^{(i + 1)}, \dots, \mathcal{C}_0^{(i + 1)} \)).

To obtain an ordering of nodes of \( G_0 \) based on the usefulness of their prolongation, the embedding \( \Psi_{i + 1} \) is fully prolonged to a temporary embedding of the full graph, \( \Psi_0^\mathrm{temp} \). A downstream model is then trained using this temporary embedding to obtain \( \mathmat{Y}_\mathrm{pred} \), the predicted posterior distribution of classes for each node in \( G_0 \) (e.g. the output of the softmax layer of an MLP). The entropy of this distribution is measured, representing the amount of uncertainty in the classifier for each given node. The nodes are ordered based on the entropy from highest to lowest. This reflects the principle that it is most useful to prolong those nodes where the downstream classifier is the least certain. For downstream tasks other than node classification, the ordering would need to be defined in a different manner (for example using labels, which are not available for all nodes in our case), however the approach of prolonging the nodes about which the downstream model is the most uncertain can be extended to other tasks.
